\documentclass[]{article}
\usepackage{lmodern}
\usepackage{amssymb,amsmath}
\usepackage{ifxetex,ifluatex}
\usepackage{fixltx2e} % provides \textsubscript
\ifnum 0\ifxetex 1\fi\ifluatex 1\fi=0 % if pdftex
  \usepackage[T1]{fontenc}
  \usepackage[utf8]{inputenc}
\else % if luatex or xelatex
  \ifxetex
    \usepackage{mathspec}
  \else
    \usepackage{fontspec}
  \fi
  \defaultfontfeatures{Ligatures=TeX,Scale=MatchLowercase}
\fi
% use upquote if available, for straight quotes in verbatim environments
\IfFileExists{upquote.sty}{\usepackage{upquote}}{}
% use microtype if available
\IfFileExists{microtype.sty}{%
\usepackage{microtype}
\UseMicrotypeSet[protrusion]{basicmath} % disable protrusion for tt fonts
}{}
\usepackage[margin=1in]{geometry}
\usepackage{hyperref}
\hypersetup{unicode=true,
            pdftitle={R Notebook},
            pdfborder={0 0 0},
            breaklinks=true}
\urlstyle{same}  % don't use monospace font for urls
\usepackage{color}
\usepackage{fancyvrb}
\newcommand{\VerbBar}{|}
\newcommand{\VERB}{\Verb[commandchars=\\\{\}]}
\DefineVerbatimEnvironment{Highlighting}{Verbatim}{commandchars=\\\{\}}
% Add ',fontsize=\small' for more characters per line
\usepackage{framed}
\definecolor{shadecolor}{RGB}{248,248,248}
\newenvironment{Shaded}{\begin{snugshade}}{\end{snugshade}}
\newcommand{\KeywordTok}[1]{\textcolor[rgb]{0.13,0.29,0.53}{\textbf{#1}}}
\newcommand{\DataTypeTok}[1]{\textcolor[rgb]{0.13,0.29,0.53}{#1}}
\newcommand{\DecValTok}[1]{\textcolor[rgb]{0.00,0.00,0.81}{#1}}
\newcommand{\BaseNTok}[1]{\textcolor[rgb]{0.00,0.00,0.81}{#1}}
\newcommand{\FloatTok}[1]{\textcolor[rgb]{0.00,0.00,0.81}{#1}}
\newcommand{\ConstantTok}[1]{\textcolor[rgb]{0.00,0.00,0.00}{#1}}
\newcommand{\CharTok}[1]{\textcolor[rgb]{0.31,0.60,0.02}{#1}}
\newcommand{\SpecialCharTok}[1]{\textcolor[rgb]{0.00,0.00,0.00}{#1}}
\newcommand{\StringTok}[1]{\textcolor[rgb]{0.31,0.60,0.02}{#1}}
\newcommand{\VerbatimStringTok}[1]{\textcolor[rgb]{0.31,0.60,0.02}{#1}}
\newcommand{\SpecialStringTok}[1]{\textcolor[rgb]{0.31,0.60,0.02}{#1}}
\newcommand{\ImportTok}[1]{#1}
\newcommand{\CommentTok}[1]{\textcolor[rgb]{0.56,0.35,0.01}{\textit{#1}}}
\newcommand{\DocumentationTok}[1]{\textcolor[rgb]{0.56,0.35,0.01}{\textbf{\textit{#1}}}}
\newcommand{\AnnotationTok}[1]{\textcolor[rgb]{0.56,0.35,0.01}{\textbf{\textit{#1}}}}
\newcommand{\CommentVarTok}[1]{\textcolor[rgb]{0.56,0.35,0.01}{\textbf{\textit{#1}}}}
\newcommand{\OtherTok}[1]{\textcolor[rgb]{0.56,0.35,0.01}{#1}}
\newcommand{\FunctionTok}[1]{\textcolor[rgb]{0.00,0.00,0.00}{#1}}
\newcommand{\VariableTok}[1]{\textcolor[rgb]{0.00,0.00,0.00}{#1}}
\newcommand{\ControlFlowTok}[1]{\textcolor[rgb]{0.13,0.29,0.53}{\textbf{#1}}}
\newcommand{\OperatorTok}[1]{\textcolor[rgb]{0.81,0.36,0.00}{\textbf{#1}}}
\newcommand{\BuiltInTok}[1]{#1}
\newcommand{\ExtensionTok}[1]{#1}
\newcommand{\PreprocessorTok}[1]{\textcolor[rgb]{0.56,0.35,0.01}{\textit{#1}}}
\newcommand{\AttributeTok}[1]{\textcolor[rgb]{0.77,0.63,0.00}{#1}}
\newcommand{\RegionMarkerTok}[1]{#1}
\newcommand{\InformationTok}[1]{\textcolor[rgb]{0.56,0.35,0.01}{\textbf{\textit{#1}}}}
\newcommand{\WarningTok}[1]{\textcolor[rgb]{0.56,0.35,0.01}{\textbf{\textit{#1}}}}
\newcommand{\AlertTok}[1]{\textcolor[rgb]{0.94,0.16,0.16}{#1}}
\newcommand{\ErrorTok}[1]{\textcolor[rgb]{0.64,0.00,0.00}{\textbf{#1}}}
\newcommand{\NormalTok}[1]{#1}
\usepackage{graphicx,grffile}
\makeatletter
\def\maxwidth{\ifdim\Gin@nat@width>\linewidth\linewidth\else\Gin@nat@width\fi}
\def\maxheight{\ifdim\Gin@nat@height>\textheight\textheight\else\Gin@nat@height\fi}
\makeatother
% Scale images if necessary, so that they will not overflow the page
% margins by default, and it is still possible to overwrite the defaults
% using explicit options in \includegraphics[width, height, ...]{}
\setkeys{Gin}{width=\maxwidth,height=\maxheight,keepaspectratio}
\IfFileExists{parskip.sty}{%
\usepackage{parskip}
}{% else
\setlength{\parindent}{0pt}
\setlength{\parskip}{6pt plus 2pt minus 1pt}
}
\setlength{\emergencystretch}{3em}  % prevent overfull lines
\providecommand{\tightlist}{%
  \setlength{\itemsep}{0pt}\setlength{\parskip}{0pt}}
\setcounter{secnumdepth}{0}
% Redefines (sub)paragraphs to behave more like sections
\ifx\paragraph\undefined\else
\let\oldparagraph\paragraph
\renewcommand{\paragraph}[1]{\oldparagraph{#1}\mbox{}}
\fi
\ifx\subparagraph\undefined\else
\let\oldsubparagraph\subparagraph
\renewcommand{\subparagraph}[1]{\oldsubparagraph{#1}\mbox{}}
\fi

%%% Use protect on footnotes to avoid problems with footnotes in titles
\let\rmarkdownfootnote\footnote%
\def\footnote{\protect\rmarkdownfootnote}

%%% Change title format to be more compact
\usepackage{titling}

% Create subtitle command for use in maketitle
\newcommand{\subtitle}[1]{
  \posttitle{
    \begin{center}\large#1\end{center}
    }
}

\setlength{\droptitle}{-2em}
  \title{R Notebook}
  \pretitle{\vspace{\droptitle}\centering\huge}
  \posttitle{\par}
  \author{}
  \preauthor{}\postauthor{}
  \date{}
  \predate{}\postdate{}


\begin{document}
\maketitle

\begin{Shaded}
\begin{Highlighting}[]
\NormalTok{anno <-}\StringTok{ }\KeywordTok{read.delim}\NormalTok{(}\DataTypeTok{header=}\OtherTok{TRUE}\NormalTok{, }\DataTypeTok{file=}\StringTok{'/home/jdm/ProjectOWE11/WCFS1_anno.txt'}\NormalTok{, }\DataTypeTok{skip=}\DecValTok{1}\NormalTok{)}
\NormalTok{counts <-}\StringTok{ }\KeywordTok{read.delim}\NormalTok{(}\DataTypeTok{header=}\OtherTok{TRUE}\NormalTok{, }\DataTypeTok{file=}\StringTok{'/home/jdm/ProjectOWE11/RNA-Seq-counts.txt'}\NormalTok{)}

\KeywordTok{head}\NormalTok{(anno)}
\end{Highlighting}
\end{Shaded}

\begin{verbatim}
##       ORF start stop orientation    name
## 1 lp_0001     1 1365           +    dnaA
## 2 lp_0002  1546 2682           +    dnaN
## 3 lp_0004  3210 3440           + lp_0004
## 4 lp_0005  3444 4565           +    recF
## 5 lp_0006  4565 6508           +    gyrB
## 6 lp_0007  6676 9234           +    gyrA
##                                           function.                 class
## 1   chromosomal replication initiation protein DnaA        DNA metabolism
## 2       DNA-directed DNA polymerase III, beta chain        DNA metabolism
## 3                                           unknown Hypothetical proteins
## 4 DNA repair and genetic recombination protein RecF        DNA metabolism
## 5                             DNA gyrase, B subunit        DNA metabolism
## 6                             DNA gyrase, A subunit        DNA metabolism
##                                    subclass       EC
## 1 DNA replication recombination, and repair         
## 2 DNA replication recombination, and repair  2.7.7.7
## 3                          Conserved: other         
## 4 DNA replication recombination, and repair         
## 5 DNA replication recombination, and repair 5.99.1.3
## 6 DNA replication recombination, and repair 5.99.1.3
##   LocateP.Subcellular.Localization.Prediction  X X.1 X.2 X.3
## 1                              @Intracellular NA  NA  NA    
## 2                              @Intracellular NA  NA  NA    
## 3                              @Intracellular NA  NA  NA    
## 4                              @Intracellular NA  NA  NA    
## 5                              @Intracellular NA  NA  NA    
## 6                              @Intracellular NA  NA  NA
\end{verbatim}

\begin{Shaded}
\begin{Highlighting}[]
\KeywordTok{head}\NormalTok{(counts)}
\end{Highlighting}
\end{Shaded}

\begin{verbatim}
##        ID WCFS1.glc.1 WCFS1.glc.2 WCFS1.rib.1 WCFS1.rib.2 NC8.glc.1
## 1 lp_0001        8100        9599        8144        7000      7117
## 2 lp_0002       12679       15856       11539       11049     10815
## 3 lp_0004        1795        1946        1470        1607      1489
## 4 lp_0005        8538        8740        5699        7402      6497
## 5 lp_0009       56040       42130       31941       23500     61965
## 6 lp_0010      105615       90094       60086       52584    103873
##   NC8.glc.2 NC8.rib.1 NC8.rib.2
## 1      8278      7457      6980
## 2     14348     10552     10735
## 3      1407      1587      1699
## 4      8565      6581      8342
## 5     37353     20498     18188
## 6     91726     45530     44802
\end{verbatim}

\begin{Shaded}
\begin{Highlighting}[]
\CommentTok{# take a slice from df and change colnames}
\NormalTok{anno <-}\StringTok{ }\NormalTok{anno [,}\KeywordTok{c}\NormalTok{(}\StringTok{"ORF"}\NormalTok{,}\StringTok{"name"}\NormalTok{)]}
\KeywordTok{colnames}\NormalTok{(anno) <-}\StringTok{ }\KeywordTok{c}\NormalTok{(}\StringTok{"ID"}\NormalTok{,}\StringTok{"Gene"}\NormalTok{)}

\CommentTok{# merge two dfs on "ID"}
\NormalTok{data <-}\StringTok{ }\KeywordTok{merge}\NormalTok{(counts, anno, }\DataTypeTok{by.x=}\StringTok{"ID"}\NormalTok{, }\DataTypeTok{by.y=}\StringTok{"ID"}\NormalTok{, }\DataTypeTok{all.x=}\OtherTok{TRUE}\NormalTok{)}



\CommentTok{# write data to tsv file}
\KeywordTok{write.table}\NormalTok{(data, }\DataTypeTok{file=}\StringTok{"WCFS1_counts.tsv"}\NormalTok{, }\DataTypeTok{sep=}\StringTok{"}\CharTok{\textbackslash{}t}\StringTok{"}\NormalTok{, }\DataTypeTok{quote=}\NormalTok{F, }\DataTypeTok{row.names=}\NormalTok{F)}



\NormalTok{m2 <-}\StringTok{ }\KeywordTok{as.matrix}\NormalTok{(data[,}\DecValTok{2}\OperatorTok{:}\DecValTok{9}\NormalTok{])}

\CommentTok{# set headers }
\KeywordTok{rownames}\NormalTok{(m2) <-}\StringTok{ }\NormalTok{data[,}\StringTok{"ID"}\NormalTok{]}
\end{Highlighting}
\end{Shaded}

\begin{Shaded}
\begin{Highlighting}[]
\CommentTok{# compute correlation}
\KeywordTok{cor}\NormalTok{(m2[,}\DecValTok{1}\NormalTok{],m2[,}\DecValTok{2}\NormalTok{], }\DataTypeTok{method=}\StringTok{"spearman"}\NormalTok{)}
\end{Highlighting}
\end{Shaded}

\begin{verbatim}
## [1] 0.9939768
\end{verbatim}

\begin{Shaded}
\begin{Highlighting}[]
\KeywordTok{cor}\NormalTok{(m2[,}\DecValTok{2}\NormalTok{],m2[,}\DecValTok{3}\NormalTok{], }\DataTypeTok{method=}\StringTok{"spearman"}\NormalTok{)}
\end{Highlighting}
\end{Shaded}

\begin{verbatim}
## [1] 0.9230595
\end{verbatim}

\begin{Shaded}
\begin{Highlighting}[]
\KeywordTok{cor}\NormalTok{(m2[,}\DecValTok{4}\NormalTok{],m2[,}\DecValTok{5}\NormalTok{], }\DataTypeTok{method=}\StringTok{"spearman"}\NormalTok{)}
\end{Highlighting}
\end{Shaded}

\begin{verbatim}
## [1] 0.878491
\end{verbatim}

\begin{Shaded}
\begin{Highlighting}[]
\KeywordTok{cor}\NormalTok{(m2[,}\DecValTok{6}\NormalTok{],m2[,}\DecValTok{7}\NormalTok{], }\DataTypeTok{method=}\StringTok{"spearman"}\NormalTok{)}
\end{Highlighting}
\end{Shaded}

\begin{verbatim}
## [1] 0.9153917
\end{verbatim}

\begin{Shaded}
\begin{Highlighting}[]
\NormalTok{hc <-}\StringTok{ }\KeywordTok{hclust}\NormalTok{(}\KeywordTok{dist}\NormalTok{(}\KeywordTok{t}\NormalTok{(m2), }\DataTypeTok{method=}\StringTok{"eu"}\NormalTok{), }\DataTypeTok{method=} \StringTok{"com"}\NormalTok{)}
\KeywordTok{plot}\NormalTok{(hc)}
\end{Highlighting}
\end{Shaded}

\includegraphics{RNA-GenexpressieNotebook_files/figure-latex/unnamed-chunk-6-1.pdf}

\begin{Shaded}
\begin{Highlighting}[]
\CommentTok{# make heatmap}
\KeywordTok{heatmap}\NormalTok{(m2)}
\end{Highlighting}
\end{Shaded}

\includegraphics{RNA-GenexpressieNotebook_files/figure-latex/unnamed-chunk-7-1.pdf}

\begin{Shaded}
\begin{Highlighting}[]
\NormalTok{pr <-}\StringTok{ }\KeywordTok{prcomp}\NormalTok{(m2, }\DataTypeTok{centre=}\OtherTok{FALSE}\NormalTok{, }\DataTypeTok{scale.=}\OtherTok{FALSE}\NormalTok{)}
\end{Highlighting}
\end{Shaded}

\begin{Shaded}
\begin{Highlighting}[]
\KeywordTok{summary}\NormalTok{(pr)}
\end{Highlighting}
\end{Shaded}

\begin{verbatim}
## Importance of components:
##                              PC1       PC2       PC3       PC4       PC5
## Standard deviation     3.967e+04 9.032e+03 5.205e+03 1.759e+03 1.528e+03
## Proportion of Variance 9.314e-01 4.828e-02 1.603e-02 1.830e-03 1.380e-03
## Cumulative Proportion  9.314e-01 9.797e-01 9.957e-01 9.975e-01 9.989e-01
##                              PC6       PC7      PC8
## Standard deviation     1.140e+03 572.56208 4.75e+02
## Proportion of Variance 7.700e-04   0.00019 1.30e-04
## Cumulative Proportion  9.997e-01   0.99987 1.00e+00
\end{verbatim}

\begin{Shaded}
\begin{Highlighting}[]
\KeywordTok{biplot}\NormalTok{(}\KeywordTok{prcomp}\NormalTok{(m2), }\DataTypeTok{scale=}\NormalTok{T)}
\end{Highlighting}
\end{Shaded}

\includegraphics{RNA-GenexpressieNotebook_files/figure-latex/unnamed-chunk-10-1.pdf}


\end{document}
